\documentclass[DM,authoryear,toc]{lsstdoc}
% lsstdoc documentation: https://lsst-texmf.lsst.io/lsstdoc.html
\input{meta}

% Package imports go here.

% Local commands go here.

%If you want glossaries
%\input{aglossary.tex}
%\makeglossaries

\title{Seeing values for LSST strategy simulations}

% Optional subtitle
% \setDocSubtitle{A subtitle}

\author{%
Eric Neilsen
}

\setDocRef{RTN-022}
\setDocUpstreamLocation{\url{https://github.com/lsst/rtn-022}}

\date{\vcsDate}

% Optional: name of the document's curator
% \setDocCurator{The Curator of this Document}

\setDocAbstract{%
The \texttt{opsim4} operations simulation program for the LSST astronomical survey uses a database of seeing values covering the range of times to be simulated. I describe the creation of such a database using Dual Image Motion Monitor (DIMM) data collected at Cerro Pachon from 2004-03-17 to 2019-10-07. In times during which the data overlap, I compare the distribution of DIMM seeing values to the seeing measured in DECam images, taken at a site ~10 km away. Instrumental problems in the DIMM may indicate unreliable measurements, cuts on image quality (as indicated by the measured Strehl ratio) were explored. The DIMM has significant gaps, so I model the data (with and without cuts on Strehl ratio) and generate artificial data in the gaps according to the model. The model consists of a sinusoidal variation with a period of one year, an autoregressive (AR1) model for variations in mean seeing from one night to the next, and another AR1 model for variations on a 5 minute timescale. I create four databases according to this procedure, two based on DIMM data starting 2006-01-01 (with and without a Strehl ratio cut), and two starting 2009-01-01. I then run \texttt{opsim} simulations using each, and an otherwise identical simulation using the default seeing database, and explore the differences.
}

% Change history defined here.
% Order: oldest first.
% Fields: VERSION, DATE, DESCRIPTION, OWNER NAME.
% See LPM-51 for version number policy.
\setDocChangeRecord{%
  \addtohist{1}{YYYY-MM-DD}{Unreleased.}{Eric Neilsen}
}


\begin{document}

% Create the title page.
\maketitle
% Frequently for a technote we do not want a title page  uncomment this to remove the title page and changelog.
% use \mkshorttitle to remove the extra pages

% ADD CONTENT HERE
% You can also use the \input command to include several content files.

\appendix
% Include all the relevant bib files.
% https://lsst-texmf.lsst.io/lsstdoc.html#bibliographies
\section{References} \label{sec:bib}
\renewcommand{\refname}{} % Suppress default Bibliography section
\bibliography{local,lsst,lsst-dm,refs_ads,refs,books}

% Make sure lsst-texmf/bin/generateAcronyms.py is in your path
\section{Acronyms} \label{sec:acronyms}
\addtocounter{table}{-1}
\begin{longtable}{p{0.145\textwidth}p{0.8\textwidth}}\hline
\textbf{Acronym} & \textbf{Description}  \\\hline

2D & Two-dimensional \\\hline
DE & dark energy \\\hline
DES & Dark Energy Survey \\\hline
DIMM & Differential Image Motion Monitor \\\hline
DM & Data Management \\\hline
FWHM & Full Width at Half-Maximum \\\hline
L1 & Lens 1 \\\hline
LSST & Legacy Survey of Space and Time (formerly Large Synoptic Survey Telescope) \\\hline
MJD & Modified Julian Date (to be avoided; see also JD) \\\hline
NOAO & National Optical Astronomy Observatories (USA) \\\hline
PSF & Point Spread Function \\\hline
RA & Right Ascension \\\hline
RMS & Root-Mean-Square \\\hline
RTN & Rubin Technical Note \\\hline
WFD & Wide Fast Deep \\\hline
\end{longtable}

% If you want glossary uncomment below -- comment out the two lines above
%\printglossaries





\end{document}
